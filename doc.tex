\documentclass[fullpage]{report}
\usepackage[margin=1in]{geometry}
\begin{document}


 \title{DPU WormTracker}
 \author{Valerie Simonis}
% \maketitle


% \tableofcontents
% \chapter{Problem Description}

\chapter{Installation}
\section{GitHub}
\subsection{Setup}
\begin{tabbing}
\verb|sudo apt-get install git|~~~~~~~~~~~~~~~~~~~~~~~~~~\=Install git \\
\verb|git init|\>Setup as git capable repository \\
\verb|git config --global user.name "name"|\>Configure machine to have an author\\
\verb|git config --global user.email "email"|\>Configure machine to have an author\verb|git remote -v|~~~~~~~~~~~~~~~~~~~~~~~~~~~~~~~\=List local remotes \\
\end{tabbing}


\section{Emacs Shortcuts}
\subsection{Motion}
\begin{tabbing}
line:~~~~~~~~~\=\verb|C-a|~~~~~~\=\verb|C-e|~~~~~~\=(beginning/end)\\
character:\>\verb|C-f|\>\verb|C-b|\>(forward/backward)\\
word:\>\verb|ESC-f|\>\verb|ESC-b|\>(forward/backward)\\
\end{tabbing} 


\chapter{Current System}
% \section{Hardware}


\section{Software}
\subsection{Techniques}
These are some of the general techniques that are being used in the structure/implementation of the \verb|WormTracker|
\begin{enumerate}
\item Logging
\item Passing execution parameters through the command line
\item Qt (User Interface)
\item GitHub for versioning and code distribution. Current branch being used is iss20
\end{enumerate}
\subsection{Classes/Files}
\begin{itemize}
\item \verb|install-opencv-2_4_8.sh|: shell script to have identical OpenCV installations across machines
\item \verb|tracker.py|: the master class executed for functionality of tracker. Defines \verb|Tracker| class. Initializes Qt Windows, passes arguments to the \verb|finder|.
\item \verb|easyEBB.py|: management class for the EiBotBoard (motor control board)
\item \verb|eggbot_scanlinux.py|: class used to search USB devices connected for the EiBotBoard. Modified from Inkscape Python extension for EiBotBoard.
\item \verb|finder.py|: class containing all of the methods used to locate the worm in an image and make decisions about whether or not it needs to move. 
\item \verb|managers.py|: management class for the Qt display and recording
\item \verb|imgProc.py|: depreciated


\end{itemize}

\subsection{Command line arguments}
NOTE: some of the execution methods and logging levels have certain UI behaviors attached to them. \\
\noindent
Sample execution of the tracker program is as follows: \vspace{5 mm}
\verb|sudo python tracker.py -m lazyc -l warning -s 0|\\
\vspace{5mm}\noindent
NOTE: \verb|sudo| needed to be able to interact with the stepper motors

\subsubsection{Source (-s)}
These are numbered as follows:
\begin{enumerate}
\setcounter{enumi}{-1}
\item Camera source 0 (webcam on laptop or USB camera on Intel NUC)
\item Camera source 1 (USB camera on laptop)
\item \verb|led_move1.avi|
\item \verb|screencast.avi|
\item \verb|shortNoBox.avi|
\item \verb|longNoBox.avi|
\item \verb|H299.avi|
\end{enumerate}
\noindent
A sources 2-7 are useful to test execution of tracker without using live worms. An improvement in the tracker from a previous version can be obvious when running it on 

\subsubsection{Execution Methods (-m)}
These are 4-5 letter specifications (command line \verb|-m| flag) specifying certain elements of operation of the  \vspace{5 mm}
\begin{enumerate}
\item \verb|surf|
\item \verb|sift|
\item \verb|lazy|
\item \verb|lazyc|
\item \verb|lazyd|
\item \verb|full|
\item \verb|test|
\end{enumerate}

The default method is specified line 208 in \verb|tracker.py| as is currently \verb|lazy|

\subsubsection{Logging Levels (-l)}
These are standard and will impact what information is seen in the console. There are also UI considerations and dependencies on the execution method. 
\begin{itemize}
\item \verb|critical| associated with value 50. 
\item \verb|error|
\item \verb|warning|
\item \verb|info|
\item \verb|debug|
\end{itemize}

\subsubsection{Recommended combinations of -m -l and -s depending on goals}

\chapter{Improvements}
\section{Hardware}
\section{Software}
Overall important things to change:
\begin{enumerate}
\item Make the system multithreaded. The findworm
\end{enumerate}
\subsection{findworm}
\subsection{Interactions with hardware}
It is important to remember that any time the stepper motors are engaged there is about a 1s recovery time to return to the 
\begin{enumerate}
\item Camera: due to OpenCV issues with the camera delivering video in \verb|GRAY-8| part of the OpenCV C class 
\end{enumerate}


\end{document}